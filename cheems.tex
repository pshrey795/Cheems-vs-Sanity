\documentclass{article}
\usepackage[utf8]{inputenc}
\usepackage{graphicx}
\usepackage{geometry}
\usepackage{hyperref}
\geometry{a4paper, total={170mm,257mm}, left=30mm, top=10mm}

\begin{document}

    \tableofcontents

    \begin{abstract}
        Nowadays, memes are very popular throughout the Internet. They are used to express feelings, like happiness, anger, or sadness. They have become one of the most dominant forms of communication in the digital age and possibly the most favourite pastime of the younger generation. The beauty of a meme is that it is not just a picture, but a story. The story is the main content of the meme, and the story is the main reason for the popularity of the meme. And another feature of memes, that results from the larger context of the memes,  is that more often than not, they are interpreted differently by different people. Due to this, sometimes, even a single meme, which is nothing more than a piece of image or text, becomes a recurrent topic of discussion, as evident from the thousands of comments following some of the more popular memes on Reddit or Instagram. 

        Following this reasoning, we will try and analyze the effect memes have on language. We use the meme language \textit{\textbf{Cheems}} for our case study. We intend to delve deep in the origin, history, and usage of the language and also try to understand the usage of the language in the context of the memes. We will also discuss the evolution of the language among the community using various conversational examples and try to put forth various opinions of different language groups on the validity of the language.
    \end{abstract}

    \section{Theoritical Framework}

    \section{Examples from Data}

    \section{Conclusion}

    \section{Notes and References}
\end{document}