\documentclass{article}
\usepackage[utf8]{inputenc}
\usepackage{graphicx}
\usepackage{geometry}
\usepackage{hyperref}
\geometry{a4paper, total={170mm,257mm}, left=30mm, top=10mm}

\begin{document}

    \tableofcontents

    \begin{abstract}
        Nowadays, memes are very popular throughout the Internet. They are used to express feelings, like happiness, anger, or sadness. They have become one of the most dominant forms of communication in the digital age and possibly the most favourite pastime of the younger generation. The beauty of a meme is that it is not just a picture, but a story. The story is the main content of the meme, and the story is the main reason for the popularity of the meme. And another feature of memes, that results from the larger context of the memes,  is that more often than not, they are interpreted differently by different people. Due to this, sometimes, even a single meme, which is nothing more than a piece of image or text, becomes a recurrent topic of discussion, as evident from the thousands of comments following some of the more popular memes on Reddit or Instagram. 

        Following this reasoning, we will try and analyze the effect memes have on language. We use the meme language \textit{\textbf{Cheems}} for our case study. We intend to delve deep in the origin, history, and usage of the language and also try to understand the usage of the language in the context of the memes. We will also discuss the evolution of the language among the community using various conversational examples and try to put forth various opinions of different language groups on the validity of the language.

        The above discussion is significant because it reflects on the development of languages in short term, and highlights the importance of digital media in this day and age and the impact it has shown on our community as a whole. We will try to show that the technology of today has played a pivotal role in truly turning the world into a global village, and we spec
    \end{abstract}

    \section{Theoritical Framework}
    Normally, a particular language develops and evolves over hundreds and possibly thousands of years, for it to seep through a wide and diverse range of language communities. Most of the traditional languages like Hindi, Gujarati, English, etc. have been spread and passed upon, more or less through direct human communication and have been used in the past for a long time.  The prevalence which these languages have achieved is primarily because it has been passed and propogated through a variety of media like human speech and interaction, writing in the form of print media. But note that the biggest hindrance to the rapid spread of these languages is the sense of locality. The geographical barriers 

    \section{Examples from Data}
    \begin{enumerate}
        \item $rishi \rightarrow ri+m+shi = rimshi$
        \item $sayam \rightarrow sa+m+yam = samyam$
        \item $shrey \rightarrow s+m+rey = smrey$
    \end{enumerate}
    Conversion between the three peers:
    \begin{enumerate}
        \item $Rimshi$: Hemlo Smrey, how are you?
        \item $Smrey$: Hemlo Rimshi, I am goomd. Are you in campus?
        \item $Rimshi$: Yemsyesm, I came yemsterday. \\
        \\
        Meanwhile Samyam joins the conversation.
        \item $Samyam$: Common you gumys, are'nt you forgemtting something!
        \item $Rimshi$ and $Smrey$: What?
        \item $Samyam$: We hame a deamdline for the term pamper tomday. Hame you even demcided the tompic yet?
        \item $Smrey$: Yemsyesm, the term paper is for lamguage course. So whmy mot write ambout our namtive lamguage.
        \item $Rimshi$: Smrey, are you remferring to Cheems? 
        \item $Smrey$: Examtly!
        \item $Samyam$: Oh yeah! Cheems is just not a lamguage, its an emotion for us. Brimiant imdea Smrey.
    \end{enumerate}

    \section{Problems of Fit}

    \section{Conclusion}

    \section{Notes and References}
\end{document}