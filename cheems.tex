\documentclass{article}
\usepackage[utf8]{inputenc}
\usepackage{graphicx}
\usepackage[a4paper, total={170mm,257mm}, left=25mm, top=20mm, right=25mm, bottom=20mm]{geometry}
\usepackage{indentfirst}
\usepackage{soul}
\usepackage{float}
\usepackage{enumitem}
\usepackage[colorlinks=true, linkcolor=blue, urlcolor=cyan]{hyperref}

\title{Cheems vs Sanity}
\author{Sayam Sethi (2019CS10399)\\Shrey Patel (2019CS10400)\\Rishi Shah (2019CS10394)}
\date{November, 2021}

\begin{document}

\maketitle

\tableofcontents

\newpage

\begin{abstract}
    Nowadays, memes are very popular throughout the Internet. They are used to express feelings, like happiness, anger, or sadness. They have become one of the most dominant forms of communication in the digital age and possibly the most favourite pastime of the younger generation. The beauty of a meme is that it is not just a picture, but a story. The story is the main content of the meme, and the story is the main reason for the popularity of the meme. And another feature of memes, that results from the larger context of the memes,  is that more often than not, they are interpreted differently by different people. Due to this, sometimes, even a single meme, which is nothing more than a piece of image or text, becomes a recurrent topic of discussion, as evident from the thousands of comments following some of the more popular memes on Reddit or Instagram.

    Following this reasoning, we will try and analyze the effect memes have on language. We use the meme language \textbf{Cheems} for our case study. We intend to delve deep in the origin, history, and usage of the language and also try to understand the usage of the language in the context of the memes. We will also discuss the evolution of the language among the community using various conversational examples and try to put forth various opinions of different language groups on the validity of the language.

    The above discussion is significant because it reflects on the development of languages in short term, and highlights the importance of digital media in this day and age and the impact it has shown on our community as a whole. We will try to show that the technology of today has played a pivotal role in truly turning the world into a global village, and we speculate that this study will give us an insight on the future development of languages or atleast what we can expect going forward.
\end{abstract}

\section{Theoretical Framework}

\subsection{Introduction}
Normally, a particular language develops and evolves over hundreds and possibly thousands of years, for it to seep through a wide and diverse range of language communities. Most of the traditional languages like Hindi, Gujarati, English, etc. have been spread and passed upon, more or less through direct human communication and have been used in the past for a long time.  The prevalence which these languages have achieved is primarily because it has been passed and propogated through a variety of media like human speech and interaction, writing in the form of print media. But note that the biggest hindrance to the rapid spread of these languages is the sense of locality. The geographical barriers have caused the language spread to be abysmally slow and gradual.

However, the digital media today has caused information to flow much faster, regardless of barriers. So, nowadays, people are quicker to adapt to various cultural (especially related to language) and technological trends and concepts as compared to the past. People today are much more likely to be aware of affairs and incidents happening in different parts of the world. The teenagers today use a whole array of web jargon which supposedly makes them appear "cool" among their peers, like \textit{lol} (laugh out loud), \textit{gg} (good game), \textit{brb} (be right back) and so on. And what's fascinating is that all these web jargon then becomes a part of their daily language as well. It is not uncommon to come across who prefers to actually speak \textit{lol} instead of actually laughing at some joke or humorous remark. Another surprising fact about these internet "lingo" (pun totally intended) is that its users don't actually go through any sort of training or learning process. It's just practice through usage. No rules, no bounds, only usage. The surprising fluency may be because these are generally derived from a parent language, mostly English, but may be extended to other languages as well or due to the fact that the users of this language are quite frequently exposed to the usage of this kind of abbreviated language in different parts of the net. The constant exposure seems to be the better reason for the rapid growth in its users.

Memes play a kind of a unique role in propogating this information. One might ask that online educational portals, blogs or research papers that are freely available throughtout the Internet should be the primary sources of intellectual ideas and concepts, and they are, to a large extent, but the painful reality is that most of us are often not patient or curious enough to dig through scores of \st{bland} text to get our daily dose of brain food. Most of us won't even log onto the Internet intending to learn something new but just to have fun or kill some time. Memes come to rescue in this scenario. They are fun, witty, and informative (mostly). Many times, they mock some stereotyped belief using hilarious examples or situations, or depict raging and controversial topics in a light-hearted manner.

But the general view among the \st{boomers} digitally innocent generation is that the memes are a useless waste of time, that fill the brain with twisted and unconventional ideals and that they are another one of those undesirable outcomes when you give the youngsters too much freedom of speech and expression. But, memes are much more than just mere social media posts, they are a part of digital culture. Let's take an example. Consider the following meme:

\begin{figure}[H]
    \centering
    \includegraphics[width=0.4\textwidth]{figures/meme1.png}
    \caption{A typical Internet meme}
\end{figure}

Now, the above meme has two visible humorous elements. One is exploiting the similarity between the spellings of ac-tor and doc-tor to create a sense of contrast. And the other, more intellectual one, is the satire on Indian education in which the meme refers to the situation in which a child conveys his dream/wish of becoming an actor to his father, but the stereotypical Indian father only considers being a doctor (or an engineer :) ) to be a good job, since they are usually the most respected/highly paid jobs in the country.So, in a satirical way, this meme portrays the state of Indian education and economy, though exaggerated, to the rest of the world, since the Indian culture has historically been the one which signifies hard work, skill, progress and innovation. However, this doesn't mean that the Indian culture has ever undermined or ignored talents in other fields like art, music, design, etc.
\subsection{Origin and Etymology of Cheems}
Before \textbf{Cheems} became famous, a certain set of memes called \textbf{doge memes} rose to prominence. The word \textbf{doge} is a misspelling of the word \textit{dog}. One of the initial versions of the meme is:
\begin{figure}[H]
    \centering
    \includegraphics[width=0.5\textwidth]{figures/original_doge_meme.jpg}
    \caption{Original Doge Meme}
\end{figure}

As can be clearly seen from the meme, even though the language is still English, the way the phrases are structured gives it a very different feeling. Additionally, the phrases are terse and they don't follow the regular English grammar. In a way, the words use their root form instead of using the contextual morphemes. Another example of a meme of this category is:
\begin{figure}[H]
    \centering
    \includegraphics[width=0.3\textwidth]{figures/doge_2.jpg}
    \caption{Another example of the \textit{doge meme}}
\end{figure}

Again as we can see, the meme aims to convey a story without using too many words and it has the \textit{doge} flavour of English. The story being narrated is as follows:
\begin{itemize}
    \item The dog is requesting its master to not do something
    \item We then find out that the dog is requesting the master to trim its nails the next day
    \item The dog conveys that it is scared
    \item They also wonder why the master has decided to trim their nails
    \item They conclude that the master is trimming their nails because they scratched someone and hence promise that they will never scratch again
    \item The meme then ends with the dog requesting again
\end{itemize}


\subsubsection{Origin of the Doge Dog}
The dog in the original meme has been traced back to a female dog of the breed \textit{Shiba Inu} called \textbf{Kabosu}. On February 13th, 2010, Japanese kindergarten teacher Atsuko Sato posted several photos of her rescue-adopted Shiba Inu dog Kabosu to her personal blog. Among the photos included a peculiar shot of Kabosu sitting on a couch while glaring sideways at the camera with raised eyebrows. This is how the stock meme image originated. The first meme came to light in July 2013. The meme then went on to become the top meme of 2013 by \textbf{Know Your Meme}.\\
The popularity of the doge meme also led to creation of a cryptocurrency being named \textbf{Dogecoin}. The cryptocurrency was created by a team of developers who were inspired by the doge meme. \textbf{Dogecoin} was officially launched on December 6, 2013.
\begin{figure}[H]
    \centering
    \includegraphics[width=0.3\textwidth]{figures/Dogecoin_logo.png}
    \caption{Dogecoin Logo}
\end{figure}

\subsubsection{Shift from Doge to Cheems}
Since the introduction of the Doge meme, there have been various spin-offs and \textbf{Cheems} is one of them. On September 4th, 2017, Instagram user \href{https://www.instagram.com/balltze}{@balltze} posted an image of a Shiba Inu. The image of the dog's head was later used in ``Cheems'' memes. The earliest known usage of ``Cheemsburbger'' was published on the /r/dogelore subreddit on June 8th, 2019. This is what inspired the name \textbf{Cheems} and the language of `m'. One of the initial meme is presented here:
\begin{figure}[H]
    \centering
    \includegraphics[width=0.4\textwidth]{figures/cheems_1.jpg}
    \caption{One of the first Cheems meme hinting at the modified language}
\end{figure}

The meme further evolved into the \textbf{Buff Doge vs. Cheems} meme in which representatives of the same group from two historical eras are presented as Swole Doge and Cheems and are compared to each other. One such meme is as follows:
\begin{figure}[H]
    \centering
    \includegraphics[width=0.5\textwidth]{figures/cheems_2.jpg}
    \caption{Buff Doge vs Cheems}
\end{figure}

As can be seen, this meme depicts the generation gap and the shift from the \textit{Doge} to the \textit{Cheems} generation. Most memes similar to this one, aim to show the increase in the \textit{luxurious} lifestyle of the people with time and hence they have become \textit{weaker} and highly dependent on the excessive amenities given to them. Additionally, they show how the newer generation has become less tolerant to the various adversities in life and hence they are more prone to depression and anxiety.

\subsection{Transition from Meme to Language}
As we mentioned above, the Cheems language started as a meme and as we might assume in normal circumstances, memes are generally free-form, in the sense that they don't follow any pre-determined structure or grammar rules, while on the other hand, any widely spoken language (however arbitrary) tends to follow some rules or basic structure, some of which are decided by common universal logic (on the basis of pronounciation, meaning, etc.), while othe rules may be enforced implicitly by persistent usage. We present some of those rules here. However, note that most of these rules are not concrete, since they are based on observations and experiences of usage of this language.

\begin{itemize}
    \item Cheems is an infant language which initially started with a few fun-sounding words like \textit{dom't, birmd, thamk}, etc. It's not difficult to notice that all of these words are derived from English words by adding the letter 'm' at appropriate places. This is the only rule which is concrete, since it is the identity of the language itself. All the other rules that we list next are speculations based on observations.

    \item \underline{\textbf{Rules for adding m}}
          \begin{itemize}
              \item The letter `m' is usually added (if added) in the coda of any particular syllable of the corresponsing English word. And most of the times, it is added right after the vowel of the nucleus, as in $bad \to bamd$, where the letter `m' is added after the vowel ``a'' in the coda ``md''.
              \item In cases where the coda has multiple potential places of insertion, like in \textit{bird}, the place of insertion of "m" depends upon how much sense the resulting pronounciation makes after adding `m'. For example, pronouncing ``birmd'' is easier than pronouncing ``bimrd'' or ``birdm''.
                    %Add "I domt know, I am just a birmd"

              \item If , multiple places of insertion of "m" result in similar pronounciations, then we (this is a completely personal rule) use both the words in conjunction. For example, in the word \textit{yes}, we can insert "m" at two different places to obtain two Cheems variants of \textit{yes}, i.e. \textit{yems} by adding "m" before "s" or \textit{yesm} by adding "m" after "s" and both these words can be pronounced with same degree of ease. So, instead of using only one of them, we use \textit{yemsyesm} instead. Hilarious and out of logic, I know, but convenient.

              \item For certain words, it might be required to replace an existing letter in the word with "m" instead of just inserting it so as to not distort the pronounciation too much. This happens mainly in the cases where the letter "n" is present is the coda of a syllable, like the word \textit{sink}. It would be impossible to pronounce \textit{sinmk}, \textit{simnk} or \textit{sinkm} without making one of "m" or "n" silent. So, we simply replace "n" with "m" to obtain \textit{simk}. Much better, right?
          \end{itemize}
\end{itemize}

\section{Examples from Data}
\begin{itemize}
    \item $Rishi \rightarrow Ri+m+shi = Rimshi$
    \item $Sayam \rightarrow Sa+m+yam = Samyam$
    \item $Shrey \rightarrow Sh+m+rey = Shmrey$
\end{itemize}
Conversion between the three peers:
\begin{itemize}[label={}]
    \item $Rimshi$: Hemlo Shmrey, how are you?
    \item $Shmrey$: Hemlo Rimshi, I am goomd. Are you in campus?
    \item $Rimshi$: Yemsyesm, I came yemsterday.
\end{itemize}
Meanwhile, Samyam joins the conversation.
\begin{itemize}[label={}]
    \item $Samyam$: Come on you gumys, aren't you forgemtting something!
    \item $Rimshi$ and $Shmrey$: What?
    \item $Samyam$: We have a deamdline for the term pamper tomday. Have you even demcided the tompic yet?
    \item $Shmrey$: Yemsyesm, the term paper is for lamguage course. So why not write ambout our namtive lamguage.
    \item $Rimshi$: Shmrey, are you remferring to Cheems?
    \item $Shmrey$: Examtly!
    \item $Samyam$: Oh yeah! Cheems is not just a lamguage, its an emotion for us. Brimlliant imdea Shmrey.
    \item $Shmrey$: Thamk you, as always my imdeas are permfect.
    \item $Rimshi$: So whamt are you waimting for, lets starmt working on the paper.
\end{itemize}
A few moments later$\ldots$
\begin{itemize}[label={}]
    \item $Rimshi$: Gumys, have you starmted premparing for the mamjors?
    \item $Shmrey$: Oh Rimshi, sumch a foomlish question. You know I am emvereamdy for the majors.
    \item $Samyam$: I am yemt to start premparing. As always I will have to stumdy nimght before the majors.
    \item $Rimshi$: Yemsyesm, same here. However there's ome sumbject which I am comfident to score good gramdes.
    \item $Samyam$: Yeah HUL243 rimght! Just by attemding the lemctures, we get fumll clamrity of the sumbject and score goomd marks.
    \item $Shmrey$: Yemsyesm, it is also one of my famvourite sumbject.
\end{itemize}
One minute before the submission deamdline.
\begin{itemize}[label={}]
    \item $Shmrey$: Hurray! We dimd it.
    \item $Samyam$: Uff again a lamst minute submission.
    \item $Rimshi$: Yemyesm, thats the stomry of our comllege. We do emverything in the lasmt momements.
    \item All Together: Amll is wemll that emds wemll.
\end{itemize}


\section{Problems of Fit}
The language Cheems is not consistent. The framework and rules described in this paper are generally used by people of IIT Delhi, however it is not necessary that everyone follows the same set of rules. The problem with Cheems is that as it is not so well established language, every group has a different version of it. For example instead of $yemsyesm$, some prefer to just say it as $yesm$. The reason for this is that, as cheems is gaining popularity due to its easy to speak nature, people find their own way of speaking the language. Also, an important thing to note is that, whenever two $Cheems$ speaking people communicate, irrespective of their version of Cheems, the communication is proper and both can completely undertanding each others words.

\section{Conclusion}

\section{Notes and References}
% https://knowyourmeme.com/memes/doge
% https://knowyourmeme.com/memes/cheems
% https://en.meming.world/wiki/File:Swole_Doge_vs_Cheems_meme_2.jpg/

\end{document}